% Anything on a line after a `%' gets ignored by TeX
% Curly braces {, } are used for grouping

\documentclass[20pt]{article}

% This line requests non-standard packages (which you must make available)
\usepackage{amsmath,amssymb,graphicx,float}
\floatplacement{figure}{H}

\begin{document}

\begin{center}
{\large \bf Hessian Generating Manual } \\   % use \\ to end centered lines
%{\large Your name (not bold). }
\end{center}

This manual explains how to generate a complete Hessian from Alamode .fcs files. The following sections highlight the steps in order that should be taken.

%\emph{italics,} {\bf bold,} {\large larger} or {\small smaller} type.

%Subscripts$_i$ and superscripts$^{i, j}$.

%%% GENERATING YOUR STRUCTURE %%%
\section{Generating Your Structure}

Build your structure with a given lattice vector and basis vector matrix with \emph{fillbox.m}. \emph{Save these matrices for use in Section 4.}

Use writecart.m to write the positions in cartesian coordinates and writedirect.m to write the positions in direct coordinates.




%%% OBTAINING THE FCS FILE %%%
\section{Obtaining The FCS File}

Use Alamode to obtain a .fcs file as usual.


%%% OBTAINING THE NEIGHBORLISTS AND STRUCTURES%%%
\section{Obtaining the Neighborlists and Structures}

Use the neighborlist pair style \texttt{\emph{pair\_neighlist\_gen.cpp}} to generate neighborlists for your original (NLORIG) and final structures (NLNEW). You also need the original (SORIG) and new (SNEW) structures. 

NLORIG is the neighborlist of the structure used in Section 2. 

NLNEW is the neighborlist of the new structure. 

SORIG are the cartesian coordinates of the structure used in Section 2. 

SNEW are the cartesian coordinates of the new structure. 

Steps:

1) Run in.run-nl for original structure to get NEIGHLIST. Rename to NLORIG.

2) Run in.run-nl for new structure to get NEIGHLIST. Rename to NLNEW.

3) Put NLORIG, NLNEW, SORIG, SNEW into the directory containing the octave scripts.

%%% BUILDING THE HESSIAN%%%
\section{Building The Hessian}
Now we need to build the Hessian with the data obtained in Steps 1-3.

\subsection{\emph{formatfcs.m} - Building the IFC Neighborlist}

This involves taking the IFCs for the primitive lattice and applying crystal symmetry to generate IFCs for a larger supercell. After NLORIG, NLNEW, SORIG, SNEW, and the .fcs file are in the directory, run \emph{formatfcs.m}.

INPUTS: See the beginning of the \emph{formatfcs.m} file.

1) Lattice vector matrix $A$ in direct coordinates.

2) Basis vector matrix $b$ in direct coordinates.

3) Basis atoms ID vector. Obtain these by looking at the atom IDs in the Alamode .fcs file.

4) Minimum cutoff to set IFCs zero below this cutoff. See Section 4.3 for more details.


OUTPUTS:

1) RNL file.

\subsection{\emph{buildhessian.m} - Building the Hessian}
Now run \emph{buildhessian.m}

This step creates IILIST and IJLIST to be used for the hessian2 pair style.

\subsection{Setting IFCs = 0}

If you want to make IFCs = 0.0 before a certain cutoff, edit lines 134 for i-i interactions and 143 for i-j interactions.

\subsection{Taking Differences in Force Constants}
If you want to calculate the differences in IFCs between two calculations, repeat Steps 1-4 for different potentials. Then run makedifference.m to create a RNL that is a difference between the two RNLs. Then run buildhessian.m according to Step 4.



%%% TAKING DIFFERENCES IN IFCS%%%
\section{Running the TEP in LAMMPS}
See the example in the /5-run folder. The format inputs for the potential are:

1) Just make this number zero.

2) Cutoff.

3) IILIST

4) IJLIST

5) EQUIL - This is a file of equilibrium cartesian positions created in Step 1.


\end{document}
\grid

